\section{Conclusion}
The main objective of this thesis was to perform a semantic segmentation of land use with deep learning methods. For this purpose, three renowned reference architectures have been studied thoroughly. Each architecture was analyzed and presented in detail. The architectures were tested extensively with regards to the segmentation challenge.

The experiments proved that the W-Net architecture is ineffective on the given dataset. The unsupervised training approach generated no new insights on the data and did not lead to a adequate differentiation of class segments. For that reason, it is recommended to discard the unsupervised training setup in this specific case. However, other ways to utilize the W-Net architecture can still be explored. For example, the training could be conducted in a semi-supervised manner. This means most of the training still happens in an unsupervised environment, but a few labels are used to guide the model to the desired outcome.

By contrast, both U-Net and FC-DenseNet architectures have shown great success for the class segmentation. The predictions of U-Net were more contiguous and therefore closer to the labels used for training. On the other hand, the predictions of FC-DenseNet are closer to reality, because they represent very detailed segments. 

Both architectures expose some shortcomings with respect to the underrepresented classes in the dataset. It is recommended to address these issues before installing the models in a production environment. This is likely to be related to the concerns expressed about the inaccurate labels used for training. Therefore, it is recommended to employ a different set of labels for future research. Since manual labelling of the entire dataset comes with high efforts, another predefined set of labels has to be found. It is also possible to combine information from different sources for  this purpose.

The second objective of the thesis was to evaluate the use of SVIs to estimate the density of vegetation. It was discussed in detail to which extent the projections of the SVIs can be use to assess the suitability of areas for emergency landings. In this regard, several indices were reviewed.

Experiments with the indices have shown inconsistent results. The areas where dense vegetation is predicted differ based on the index that is used. This is unexpected, because even if the indices use different characteristics of the dataset, they should still deliver similar results. One issue with this is, that the dataset itself is lacking precise information about the vegetation itself. Thus the real vegetation density can only be estimated based on the DOPs.

Most of the vegetation indices are designed and calibrated for the use with very specific hardware under very specific conditions. The dataset used in this thesis does not match these conditions. This might explain the inconsistent projections seen across the indices.

Considering all the facts, no clear conclusion can be drawn about the use of vegetation indices. Further research is required to obtain meaningful recommendations on that topic. For that, more information about the real vegetation situation has to be considered together with the indices. By doing that, it is possible to assess which vegetation index best reflects the real vegetation conditions.

In summary, this thesis laid out the foundations to support the identification of emergency landing fields with CNNs. It is shown that semantic segmentation of land use is generally possible with that approach. Further research can build upon this thesis in order to further improve the designed methods.

\clearpage